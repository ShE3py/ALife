\documentclass[11pt]{scrartcl}

\usepackage[utf8]{inputenc}

\usepackage[T1]{fontenc}
\usepackage[french]{babel}

\usepackage{textcomp}
\usepackage{amsmath}
\usepackage{amssymb}
\usepackage{mathtools}
\usepackage{siunitx}
\usepackage{lmodern}
\usepackage{graphicx}
\usepackage{tabularx}

\usepackage{csquotes}

\usepackage{microtype}
\usepackage[parfill]{parskip}
\usepackage[hidelinks]{hyperref}

\title{Vie artificielle}
\subtitle{Croissance des organismes}

\begin{document}
    \maketitle

    \section{Introduction}\label{sec:intro}

        Le motif de Turing\footnote{\url{https://en.wikipedia.org/wiki/Turing_pattern}} est un système de
        réaction-diffusion qu'Alan Turing décrit dans son article
        \emph{The Chemical Basis of Morphogenesis}
        \footnote{\url{https://fr.wikipedia.org/wiki/The\_Chemical\_Basis\_of\_Morphogenesis}} en 1952 ;

        \textquote[Wikipédia]{Un système de réaction-diffusion est un modèle mathématique qui décrit
        l'évolution des concentrations d'une ou plusieurs substances spatialement distribuées et soumises à
        deux processus : un processus de réactions chimiques locales, dans lequel les différentes substances
        se transforment, et un processus de diffusion qui provoque une répartition de ces substances dans
        l'espace.}

        Un tel système peut être décrit par une équation différentielle partielle :
        \[
            \partial_t \, q = R(q) + \underline{D} \, \overrightarrow{\nabla}^2 q
        \]

        Où :
        \begin{itemize}
            \setlength\itemsep{.5em}
            \item $q = (q_1, q_2, \dots, q_n)$ représente la concentration chimique des $n$ substances du système ;
            \item $\partial_t \, q$ représente comment ces substances $q$ évoluent avec le temps $t$ ;
            \item $R$ représente toutes les réactions chimiques locales qui s'appliquent aux substances $q$ ;
            \item $\overrightarrow{\nabla}^2 q$ représente la diffusion des substances $q$ dans l'espace ;
            \item $\underline{D}$ représente les coefficients de cette diffusion.
        \end{itemize}

    \section{Le modèle de Gray--Scott}\label{sec:gray-scott}

        Peter Gray et S. K. Scott proposèrent en 1983 un
        modèle\footnote{\url{https://groups.csail.mit.edu/mac/projects/amorphous/GrayScott/}} revisitant celui de
        Turing\footnote{\url{https://git.litislab.fr/dolivier/VieArtificielle/-/wikis/Croissance-Des-Organismes\#morphogen\%C3\%A8se}}.
        Leur modèle contient deux espèces chimiques :

        \begin{itemize}
            \item $U$, représentant la mort ;
            \item $V$, représentant la vie.
        \end{itemize}

        Ces espèce sont réagies par deux équations chimiques :
        \begin{align}
            U + 2V &\longrightarrow 3V \label{eq:birth} \\
            V &\longrightarrow P \label{eq:decay}
        \end{align}

        L'équation~\ref{eq:birth} indique que la mort se transforme en vie au contact de la vie,
        l'équation~\ref{eq:decay} indique que la vie se dégrade périodiquement en $P$ (e.g.\ meurt sans pouvoir
        renaître).

        Cela donne le système suivant :
        \begin{gather*}
            \partial_t \, u = r_u \overrightarrow{\nabla}^2 u - uv^2 + f(1 - u) \\
            \partial_t \, v = r_v \overrightarrow{\nabla}^2 v + uv^2 - (f + k)v
        \end{gather*}

        En choisissant un pas de temps $\Delta t$, l'on peut calculer la variation des concentrations sur ce pas $\Delta t$ :
        \begin{gather*}
            \Delta u = \left(r_u \overrightarrow{\nabla}^2 u - uv^2 + f(1 - u)\right)\Delta t \\
            \Delta v = \left(r_v \overrightarrow{\nabla}^2 v + uv^2 - (f + k)v\right)\Delta t
        \end{gather*}

        Où :
        \begin{itemize}
            \setlength\itemsep{.05em}
            \item $\Delta u$ et $\Delta v$ représentent l'évolution des concentrations ;
            \item $\overrightarrow{\nabla}^2$ représente la diffusion dans l'espace ;
            \item $u$, $v$, $r$, $f$ et $k$ sont tous entre $0.0$ et $1.0$ ;
            \item $r$ représente le taux de diffusion ;
            \item $u$ et $v$ représentent les concentrations actuelles ;
            \item $f$ la vitesse la laquelle $U$ apparaît (l'on suppose que le système baigne dans une solution
                  riche en $U$) ;
            \item $k$ la vitesse de la réaction de
                  dégradation~\ref{eq:decay} (l'on ajoute $f$ pour faire que $v$ se
                  détériore plus vite que $u$ n'apparaît).
        \end{itemize}

    \section{Simulation}\label{sec:simulation}

        Le motif est simulé sur une grille de pixels ; chaque pixel a deux composantes $u$ et $v$ (entre $0.0$ et $1.0$)
        qui représentent la concentration des deux espèces chimiques.

        À chaque pas de temps, il suffit de calculer les nouvelles concentrations à partir des anciennes :
        \begin{gather*}
            u_{i + 1} = u_i + \Delta u \\
            v_{i + 1} = v_i + \Delta v
        \end{gather*}

        \subsection{Calcul de $\overrightarrow{\nabla}^2$}\label{subsec:laplace}

            Le laplacien, noté $\overrightarrow{\nabla}^2$, $\nabla^2$, ou encore $\Delta$, s'applique à
            un point et représente le gain (ou la perte) en concentration par effet de diffusion.
            On peut l'exprimer par une matrice :
            \[
                \begin{bmatrix}
                    0 & 0.25 & 0 \\
                    0.25 & -1 & 0.25 \\
                    0 & 0.25 & 0
                \end{bmatrix}
            \]

            La somme des termes est égale à $0$ (sinon on créerait/enlèverait de la concentration).
            Ici, le point perd toute sa concentration mais gagne quatre fois 25\% de la concentration de ses voisins.

        \subsection{Pseudo-code}\label{subsec:pseudo-code}

            \begin{verbatim}
float laplacian(int x, int y) {
    float sum = 0.0;

    const float MATRIX[9] = float[] (
        0.00, 0.25, 0.00,
        0.25, -1.0, 0.25,
        0.00, 0.25, 0.00
    );

    for(int dx = -1; dx <= 1; dx++) {
        for(int dy = -1; dy <= 1; dy++) {
            int offset = (3 * (dy + 1)) + (dx + 1);

            sum += grid[x + dx][y + dy].u * MATRIX[offset];
        }
    }

    return sum;
}
            \end{verbatim}

            \begin{verbatim}
// Paramètres //
float ru = 1, rv = 0.5;
float f = 0.046, k = 0.059;
float Dt = 1;

struct Cell {
    float u, v;
};

const int WIDTH = 256;
const int HEIGHT = 256;

Cell[WIDTH][HEIGHT] grid;
Cell[WIDTH][HEIGHT] nextGrid;

for ever {
    // Mise à jour des pixels
    for(int x = 0; x < WIDTH; x++) {
        for(int y = 0; y < HEIGHT; y++) {
            Cell cell = grid[x][y];
            float u = cell.u;
            float v = cell.v;

            Cell laplace = laplacian(x, y);
            float U = u + ((ru * laplace.u) - (u * v * v) + (     f  * (1.0 - u))) * Dt;
            float V = v + ((rv * laplace.v) + (u * v * v) - ((k + f) *        v )) * Dt;

            nextGrid[x][y] = Cell(U, V);
        }
    }

    // Idéalement un swap de pointeurs (pour pas recopier tout le tableau)
    swap(grid, nextGrid);

    // Dessiner `grid` à l'écran
    render();
}
            \end{verbatim}

    \section{Classification}\label{sec:classification}

        Le système est décrit par Robert Munafo avec trois paramètres\footnote{\url{https://mrob.com/sci/papers/munafo2010-0601.pdf}}
        $f$, $k$ et $\sigma = r_u / r_v$.
        L'auteur propose une classification des paramètres\footnote{\url{https://mrob.com/pub/comp/xmorphia/pearson-classes.html}}
        sur son site web, ainsi qu'une nomenclature des motifs\footnote{\url{https://mrob.com/pub/comp/xmorphia/glossary.html}} que l'on
        peut trouver.

    \section{Simulateurs}\label{sec:simulateurs}

        \begin{itemize}
            \item \url{https://pmneila.github.io/jsexp/grayscott/}
            \item \url{https://mrob.com/pub/comp/xmorphia/ogl/index.html}
            \item \url{https://itp.uni-frankfurt.de/~gros/StudentProjects/Projects_2020/projekt_schulz_kaefer/#simulation}
            \item \url{https://she3py.github.io/ALife/gray-scott.html}
        \end{itemize}

\end{document}
